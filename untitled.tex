\section{Introduction}
Embedded systems encompass a wide array of commodity hardware, from low power wearable devices, to large scale electro-mechanical systems. Network routers are an example of a very specialized embedded device, focused on facilitating digital communications. These routers range in size and power from small home network routers, to the industrial level devices that make up the backbone of the Internet. While all routers perform a common set of operations int he routing and forwarding of network data, there are drastic differences in performance, architecture, and capabilities between Core Routers and their home network counterparts.

The Internet backbone is composed of various government and Internet Service Provider (ISP) controlled network infrastructure, generally made up of Core and Edge routers. Edge routers differ from Core routers in their capacity and filtering capabilities, and are designed to provide a gateway to the core routers and the rest of the internet. This is a simplification of a very heavily layered model of network communications, composed of enterprise and consumer routers and switches working in concert to provide security and service guarantees to end users. The focus of this paper is on the hardware and software capabilities available in the routers making up the Internet Backbone. We will examine, and outline current hardware architectures and features, and the software components that are generally used in these systems. As a case study we present specifications found in a commodity Cisco router, and examine its specifications and software with the what is available in recent literature.

