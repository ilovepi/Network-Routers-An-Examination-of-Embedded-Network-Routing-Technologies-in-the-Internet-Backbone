\section{Introduction}

Embedded systems encompass a wide array of commodity hardware, from low power
wearable devices, to large scale electro-mechanical systems. Network routers
are an example of a very specialized embedded device, focused on facilitating
digital communications. These routers range in size and power from small home
network routers, to the industrial level devices that make up the backbone of
the Internet. While all routers perform a common set of operations, i.e.\ the
routing and forwarding of network data, there are drastic differences in
performance, architecture, and capabilities between Core Routers and
counterparts found in enterprise and home networks.

The Internet backbone is composed of government and commercially controlled
network infrastructure, generally made up of Core and Edge routers, as well as
their communications channels. Edge routers in many cases previously served as
core routers, but have been replaced by better performing hardware. Typically
they differ from Core routers in their capacity and filtering capabilities, and
are provide an interface to the core routers and the rest of the Internet. This
is a simplification of a heavily layered model of network communications,
composed of enterprise and consumer routers and switches working in concert to
provide security and service guarantees to end users. The focus of this paper
is on providing an embedded systems perspective for  the hardware and software
capabilities available in the routers making up the Internet Backbone.



